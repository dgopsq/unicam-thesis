\chapter{Introduzione}

La gestione del flusso dei dati all'interno di una applicazione web è un argomento molto discusso dopo l'avvento di tecnologie front-end sempre più complesse e potenti come React o Angular. La causa di ciò è la necessità di avere un codice che sia il più possibile scalabile ed il più facilmente testabile a prescindere dal numero di features che verranno successivamente aggiunte. 
Codebase vaste come potrebbero essere quelle di Facebook, Twitter o YouTube necessitano di una architettura di fondo che sia altamente chiara e comprensibile per evitare confusione tra i vari servizi.
Come vedremo successivamente, architetture datate come l'MVC pur essendo molto efficienti lato back-end non rendono allo stesso modo lato front-end, dove c'è una quantità maggiore di azioni che l'utente può intraprendere e che possono avere ripercussioni differenti su più componenti diversi all'interno di una view.
In questo documento verrà discussa l'alternativa attualmente più gettonata che è quella dell'architettura a flusso unidirezionale, implementata in prima battuta da Flux e successivamente ottimizzata da Redux.