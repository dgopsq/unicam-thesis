\chapter*{Sommario}

In questa tesi viene trattata la problematica relativa alla gestione del flusso dei dati e degli eventi che modificano direttamente o indirettamente l'interfaccia di una applicazione web moderna. Avere un adeguato controllo su questo flusso è di primaria importanza al fine di ottenere un servizio il cui codice sia chiaro e con uno stato che cambi in maniera comprensibile e deterministica, dove sia semplice riprodurre eventuali errori ed aggiungere nuovi elementi.

In questo documento verranno messe a confronto architetture innovative come \textit{Flux} e \textit{Redux} che implementano un flusso di dati unidirezionale, con altre più datate come \textit{MVC} e derivate che invece scelgono un flusso bidirezionale. 
L'analisi sarà approfondita con degli esempi di codice per ognuno di questi paradigmi in modo da effettuare una dimostrazione pratica delle loro principali differenze.