\chapter{Strumenti}
Per analizzare le varie architetture presentate dobbiamo prima fare un discorso sugli strumenti utilizzati per costruire una applicazione web e che useremo per gli esempi di codice dei capitoli successivi.

\section{ECMAScript 2015}
Anche conosciuto come \textit{ECMAScript6}, è una standardizzazione del linguaggio Javascript creata da Ecma International. Questa versione in particolare mette a disposizione features molto utili per scrivere codice funzionale. Possiamo classificare ECMAScript come un linguaggio a se e differente da Javascript, che in principio doveva essere utilizzato solamente come linguaggio di scripting lato web, ma che ora viene utilizzato come vero e proprio linguaggio di programmazione su ambienti e scale differenti \cite{ECMAScriptDocumentation}.

In ambito web non tutte le features di ES6 sono disponibili, per questo si utilizzano strumenti come \textit{Babel} o \textit{Webpack} che hanno la funzione di \textit{transpiler}, ossia di compilare codice sorgente da ECMAScript6 a ECMAScript5 che è supportato dalla stragrande maggioranza dei browser.

\section{React}
React è una libreria scritta da Facebook per la creazione di interfacce utente interattive in maniera funzionale ed altamente scalabile. Si basa sul concetto di "componente" come elemento base fondamentale, ossia un pezzo di interfaccia che ha uno stato proprio ed è riutilizzabile all'interno del servizio. Il concetto funzionale di composizione si adatta benissimo a React: un componente complesso dovrebbe essere formato da componenti più piccoli e agnostici che possono quindi essere riutilizzati in altri componenti complessi.

Viene utilizzato in produzione sia da Facebook che da Instagram e fa uso di diverse tecnologie all'avanguardia come il \textit{Virtual DOM} ed il \textit{Server-side Rendering} \cite{WheelerOnReact}.