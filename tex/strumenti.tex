\chapter{Strumenti}
Per analizzare le varie architetture presentate dobbiamo prima fare un discorso sugli strumenti utilizzati per costruire una applicazine web.

\section{ECMAScript 2015}
Anche conosciuto come \textit{ECMAScript6} (ES6), è una standardizzazione del linguaggio Javascript creata da ECMA. Questa versione in particolare mette a disposizione features estremamente utili per scrivere codice funzionale. Possiamo classificare ECMAScript come un linguaggio a se e differente da Javascript, che in principio doveva essere utilizzato solamente come linguaggio di scripting lato web, ma che ora viene utilizzato come vero e proprio linguaggio di programmazione su ambienti e scale differenti \cite{ECMAScriptDocumentation}.

In ambito web non tutte le features di ES6 sono disponibili, per questo si utilizzano strumenti come \textit{Babel} o \textit{Webpack} che hanno la funzione di \textit{transpiler}, ossia di compilare codice sorgente da ECMAScript6 a ECMAScript5 che è supportato dalla stragrande maggioranza dei browser.

