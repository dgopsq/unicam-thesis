\chapter{Strumenti}
Per analizzare le varie architetture presentate dobbiamo prima fare un discorso sugli strumenti utilizzati per costruire una applicazione web e che useremo per gli esempi di codice dei capitoli successivi.

\section{ECMAScript 2015}
Anche conosciuto come \textit{ECMAScript6}, è una standardizzazione del linguaggio Javascript creata da Ecma International. Questa versione in particolare mette a disposizione features molto utili per scrivere codice funzionale. Possiamo classificare ECMAScript come un linguaggio a se e differente da Javascript, che in principio doveva essere utilizzato solamente come linguaggio di scripting lato web, ma che ora viene utilizzato come vero e proprio linguaggio di programmazione su ambienti e scale differenti \cite{ECMAScriptDocumentation}.

In ambito web non tutte le features di ES6 sono disponibili, per questo si utilizzano strumenti come \textit{Babel} o \textit{Webpack} che hanno la funzione di \textit{transpiler}, ossia di compilare codice sorgente da ECMAScript6 a ECMAScript5 che è supportato dalla stragrande maggioranza dei browser.

\section{React}
React è una libreria scritta da Facebook per la creazione di interfacce utente interattive in maniera funzionale ed altamente scalabile. Si basa sul concetto di "componente" come elemento base fondamentale, ossia un pezzo di interfaccia che ha uno stato proprio ed è riutilizzabile all'interno del servizio. Il concetto funzionale di composizione si adatta benissimo a React: un componente complesso dovrebbe essere formato da componenti più piccoli e agnostici che possono quindi essere riutilizzati in altri componenti complessi.

Viene utilizzato in produzione sia da Facebook che da Instagram e fa uso di diverse tecnologie all'avanguardia come il \textit{Virtual DOM} ed il \textit{Server-side Rendering} \cite{WheelerOnReact}.

\subsection{Virtual DOM}
Il  \textit{Document Object Model} (DOM) è una API che definisce la struttura di un documento HTML e come essa viene acceduta e manipolata. E' una rappresentazione ad oggetti di una pagina web la quale può essere modificata con un linguaggio di scripting come Javascript.
Per fare un esempio pratico, lo standard DOM stabilisce che l'interfaccia \textit{Document} rappresenti l'intera pagina HTML e che concettualmente sia il nodo root. L'interfaccia \textit{Node} rappresenta invece l'elemento base ossia il singolo nodo all'interno di un documento e l'implementazione di questa richiede la creazione dei metodi per la gestione del nodo stesso e dei propri figli \cite{HWRWhatIsDOM}.

Quando parliamo di Virtual DOM parliamo di un'astrazione sopra l'astrazione del DOM. Modificare quest'ultimo non è particolarmente dispendioso (si tratta solamente di modificare un oggetto Javascript) è tuttavia il processo di lettura e di “ridisegno" della pagina da parte del browser il vero problema. Questa tecnologia riesce a risolvere il suddetto problema mantenendo in memoria una rappresentazione del DOM reale che utilizza il design pattern \textit{Observer}\footnotemark per capire quale particolare nodo è stato modificato generando successivamente un nuovo albero derivato dal precedente ma con il nuovo stato. A questo punto effettua complessi algoritmi di differenza per trovare il numero minimo di passaggi per aggiornare il DOM reale per poter infine effettuare la riconciliazione \cite{MishraOnVirtualDOM}.

Il concetto che permette al Virtual DOM di garantire prestazioni maggiori sul DOM reale consiste nell'aggiornamento aggregato. Tutti i cambiamenti effettuati da un evento (che sono i passaggi trovati dall'algoritmo di differenza) vengono aggregati ed il DOM viene ridisegnato solamente una volta.

React implementa il Virtual DOM attraverso \textit{JSX}, una estensione di ECMAScript simile ad XML che permette di scrivere elementi di markup con una sintassi simile all'HTML all'interno dei componenti dell'interfaccia. Portare l'HTML all'interno del codice sorgente Javascript offre dei vantaggi non banali come ad esempio il debugging compile-time degli errori di sintassi durante la costruzione del DOM, la versatilità di avere un linguaggio di scripting per effettuare composizione ed altre azioni dinamiche e sopratutto avere una perfetta separazione tra componenti differenti.

\footnotetext{Il design pattern Observer si struttura di un oggetto chiamato “Subject" che mantiene una lista di altri oggetti dipendenti chiamati "Observers" e li notifica ogni qualvolta il suo stato viene modificato.}

\subsection{Server-side rendering}
React è una libreria \textit{isomorfica}, è in grado di essere eseguita sia lato client che lato server traendo vantaggio da NodeJS ed il fatto che il principale linguaggio di programmazione per entrambi gli ambienti sia sempre Javascript.
Il vantaggio di eseguire React anche lato server risiede nella prima visualizzazione. In una Single Page Application normale durante il primo caricamento vengono scaricati gli elementi base per la sua esecuzione e successivamente viene eseguito il codice Javascript per il rendering del suo stato iniziale. La seconda fase può essere ulteriormente pesante, basti pensare che possono essere effettuate ulteriori richieste per soddisfare il normale fabbisogno dell'applicazione.

La tecnica del Server-side rendering permette di semplificare questa prima visualizzazione interpretando i componenti React lato server e restituendo una pagina iniziale con la Single Page Application già avviata e fornita di tutti i dati di cui avrebbe normalmente bisogno.